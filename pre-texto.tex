% ---
% Dedicatória
% ---
%\begin{dedicatoria}
%   \vspace*{\fill}
%   \centering
%   \noindent
%   \textit{ Este trabalho é dedicado aos estudantes apaixonados que, apesar das dificuldades, nunca deixaram de acreditar no poder transformador do conhecimento.} \vspace*{\fill}
%\end{dedicatoria}
% ---

% ---
% Agradecimentos
% ---
\begin{agradecimentos}
Primeiramente, gostaríamos de agradecer às nossas famílias, que sempre estiveram
ao nosso lado, oferecendo amor, incentivo e apoio incondicional. Aos nossos pais, por serem fontes inesgotáveis de inspiração e por nos ensinarem a importância da perseverança.
Vocês são nossa base e nossa força.
Ao nossos orientadores Maj Maurício e Cel Marcos, pela orientação meticulosa, paciência e conhecimento compartilhado ao longo deste processo. Seus conselhos e críticas construtivas foram fundamentais para moldar este trabalho.
Aos amigos que estiveram ao nosso lado, nos motivaram e entenderam nossa ausência
em tantos momentos importantes. Seus sorrisos e palavras de encorajamento foram o
combustível que nos impulsionou.
Agradecemos também a todos os professores e funcionários da instituição, que contribuíram indiretamente para a nossa formação acadêmica.
Por último, mas não menos importante, agradecemos a todos aqueles que, de uma forma
ou de outra, nos apoiaram e encorajaram durante este percurso. Cada gesto de carinho,
cada palavra amiga e cada ato de gentileza foram fundamentais para nossa motivação e
bem-estar.
% \footnote{Os nomes dos integrantes do primeiro projeto abn\TeX\ foram extraídos de \url{http://codigolivre.org.br/projects/abntex/}}

Agradecimentos especiais são direcionados aos alunos da Engenharia Cartográfica 2024 que compartilharam conosco muitas noites de estudo, debates estimulantes e momentos de desafio e que contribuíram e que sempre irão contribuir para a manuntenção do conhecimento.

\end{agradecimentos}
% ---

% ---
% Epígrafe
% ---
%\begin{epigrafe}
%    \vspace*{\fill}
%	\begin{flushright}
%		\textit{``Conhecimento é a única riqueza que, \\
%                    quanto mais compartilhada, mais cresce.\\
%		(Provérbio Chinês)}
%	\end{flushright}
%\end{epigrafe}
% ---

% ---
% RESUMOS
% ---

% resumo em português
\setlength{\absparsep}{18pt} % ajusta o espaçamento dos parágrafos do resumo
\begin{resumo}
\SingleSpacing

% Motivação
% O que foi feito
% Quais resultados são esperados ou foram alcançados
% Breve fechamento dos principais impactos do que foi feito


 \textbf{Palavras-chave}: \imprimirpalavraschave
\end{resumo}

% resumo em inglês
\begin{resumo}[Abstract]
\begin{otherlanguage*}{english}
\linespread{1.3}
\SingleSpacing
% Usem o DeepL no resumo e corrijam.
This part will be written by the VF.

%This is the english abstract.
\vspace{\onelineskip}
\noindent 
\textbf{Keywords}: \imprimirkeywords
\end{otherlanguage*}
\end{resumo}